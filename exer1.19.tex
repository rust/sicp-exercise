\documentclass{jarticle}

\usepackage{amsmath}
\begin{document}
\begin{equation}
 T_{pq} : (a, b) \rightarrow (bq + aq + ap, bp + aq)
\end{equation}
これを
\begin{equation}
 T_{pq}(a, b) = (bq + aq + ap, bp + aq)
\end{equation}
と書くと
\begin{align}
 T_{pq}(bq + aq + ap, bp + aq)
 &= T_{pq}(a', b') \\
 &= (b'q + a'q + a'p, b'p + a'q) \\
 &= ((bp + aq)q + (bq + aq + ap)q + (bq + aq + ap)p,
     (bp + aq)p + (bq + aq + ap)q) \\
 &= (bpq + aqq + bqq + aqq + apq + bpq + apq + app,
     bpp + apq + bqq + aqq + apq) \\
 &= (b(q^2 + 2pq) + a(q^2 + 2pq + p^2 + q^2),
     b(p^2 + q^2) + a(q^2 + 2pq)))
\end{align}
ここで
\begin{gather}
 p' = p^2 + q^2 \\
 q' = q^2 + 2pg
\end{gather}
とすると
\begin{align}
 T_{pq}(bq + aq + ap, bp + aq)
 &= (b(q^2 + 2pq) + a(q^2 + 2pq + p^2 + q^2),
     b(p^2 + q^2) + a(q^2 + 2pq))) \\
 &= (bq' + aq' + ap', bp' + aq')
\end{align}
と書くことができる.




\end{document}